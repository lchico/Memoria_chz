% Chapter Template

\chapter{Conclusiones} % Main chapter title

\label{Chapter5} % Change X to a consecutive number; for referencing this chapter elsewhere, use \ref{ChapterX}


%----------------------------------------------------------------------------------------

%----------------------------------------------------------------------------------------
%	SECTION 1
%----------------------------------------------------------------------------------------

\section{Conclusiones generales }

% La idea de esta sección es resaltar cuáles son los principales aportes del trabajo realizado y cómo se podría continuar. Debe ser especialmente breve y concisa. Es buena idea usar un listado para enumerar los logros obtenidos.

Al finalizar esta primera etapa del prototipo, podemos concluir que se logró:
  \begin{itemize}
    \item Mediante una página web accedemos al estado del sistema.
    \item Permite navegación mediante pc, celulares, tablets compatibles con navegadores como Chrome y Firefox.
    \item Permite realizar un control manual y automático del sistema. Mediante el uso de un sensor de temperatura y salidas para actuadores. 
    \item Alertar mediante SMS cuando: la temperatura excede el rango permitido, y/o para un determinado nivel de batería.
    \item Implementar mediante el stack TCP/IP de lwIP mediante el RTOS freeRTOS. 
    \item Uso del lenguaje C, HTML, CSS y JavaScript. 
    \item Hacer uso de un sistema de control de versiones, GitHub. 
   \end{itemize} 

Es decir, ya podemos controlar la temperatura de un tanque y recibir alertas en los principales estados críticos del proceso. Tenemos incorporado un a interfaz de monitoreo que a su vez nos permite actuar en forma remota.  



%----------------------------------------------------------------------------------------
%	SECTION 2
%----------------------------------------------------------------------------------------
\section{Próximos pasos}

%Acá se indica cómo se podría continuar el trabajo más adelante.
En esta primera etapa se dió una primera interacción con todo lo que involucra controlar el proceso de fermentación. Ahora estamos en condiciones de perfeccionar este producto extendiendo sus funciones.
Para ello podríamos optimizar los siguientes puntos:
  \begin{itemize}
    \item Ampliar la capacidad de control a 4 tanques.
    \item Mejorar la interfaz web para dichos cambios, y realizar un entorno más amigable.
    \item Ampliar las funcionalidades de los SMS, no sólo para reportes de alertas sino también para consultas y posibles controles. 
    \item Agregar retroalimentación de sensores y actuadores por hardware para un control de mas robusto.
    \item Adicionar una opción de control por intervalos de tiempos. 
   \end{itemize} 


