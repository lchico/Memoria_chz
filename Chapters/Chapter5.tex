% Chapter Template

\chapter{Conclusiones} % Main chapter title

\label{Chapter5} % Change X to a consecutive number; for referencing this chapter elsewhere, use \ref{ChapterX}


%----------------------------------------------------------------------------------------

%----------------------------------------------------------------------------------------
%	SECTION 1
%----------------------------------------------------------------------------------------

\section{Conclusiones generales }

% La idea de esta sección es resaltar cuáles son los principales aportes del trabajo realizado y cómo se podría continuar. Debe ser especialmente breve y concisa. Es buena idea usar un listado para enumerar los logros obtenidos.

Al finalizar este trabajo se logró:
  \begin{itemize}
    \item Acceder mediante una página web al estado del sistema.
    \item Navegación mediante pc, celulares, tablets compatibles con navegadores como Chrome y Firefox.
    \item Realizar un control manual y automático del sistema, mediante el uso de un sensor de temperatura y salidas para actuadores. 
    \item Alertar mediante SMS cuando la temperatura excede el rango permitido, y/o para un determinado nivel de batería.
    \item Implementar mediante el stack TCP/IP de lwIP mediante el RTOS freeRTOS. 
   \end{itemize} 

Es decir, controlar en forma remota la temperatura de un tanque y recibir alertas en los principales estados críticos del proceso.



%----------------------------------------------------------------------------------------
%	SECTION 2
%----------------------------------------------------------------------------------------
\section{Próximos pasos}

%Acá se indica cómo se podría continuar el trabajo más adelante.
Extendiendo la funcionalidad del sistema se pueden optimizar los siguientes puntos:
  \begin{itemize}
    \item Ampliar la capacidad de control a cuatro tanques.
    \item Mejorar la interfaz web para dichos cambios, y realizar un entorno más amigable.
    \item Ampliar las funcionalidades de los SMS, no sólo para reportes de alertas sino también para consultas y posibles controles. 
    \item Agregar retroalimentación de sensores y actuadores por hardware para un control de mas robusto.
    \item Adicionar una opción de control por intervalos de tiempos. 
   \end{itemize} 

