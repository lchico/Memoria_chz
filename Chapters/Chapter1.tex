% Chapter 1

\chapter{Introducción General} % Main chapter title

\label{Chapter1} % For referencing the chapter elsewhere, use \ref{Chapter1} 
\label{IntroGeneral}

%----------------------------------------------------------------------------------------

% Define some commands to keep the formatting separated from the content 
\newcommand{\keyword}[1]{\textbf{#1}}
\newcommand{\tabhead}[1]{\textbf{#1}}
\newcommand{\code}[1]{\texttt{#1}}
\newcommand{\file}[1]{\texttt{\bfseries#1}}
\newcommand{\option}[1]{\texttt{\itshape#1}}
\newcommand{\grados}{$^{\circ}$}

%----------------------------------------------------------------------------------------

\section{Introducción}

%----------------------------------------------------------------------------------------

\section{Producción de vino}


\section{Motivación}

En el presente trabajo se busca satisfacer las necesidades de la bodega de mi primo. La cual se necesita controlar un proceso delicado como es la fermentación del vino y se requiere un cuidado continúo. Por el momento es realizado por personal día y noche. Teniendo en cuenta que en la zona de trabajo es difícil conseguir personas responsables que se comprometan con el trabajo. 

A su vez se busca también una satisfacción personal, dado que hacía tiempo que quería poder implementar en un microcontrolador una interfaz que permita visualizar la información de un sistema mediante una plataforma web, donde se pueda apreciar el estado de las entradas y poder ejercer control sobre ellas.

Y para concluir, el hecho de utilizar la plataforma CIAA, la cual me permitió cumplir todo estos objetivos personales, haciendo uso de un sistema operativo de tiempo real (RTOS), como sé llamará de aquí en adelante.



\section{Proyecto CIAA}

El Proyecto CIAA cuyas siglas hacen referencia a Computadora Industrial Abierta Argentina, nació en 2013 como una iniciativa conjunta entre el sector académico y el industrial, representados por la ACSE\footnote{Asociación Civil para la investigación, promoción y desarrollo de los Sistemas electrónicos Embebidos.} y CADIEEL\footnote{Cámara Argentina de Industrias Electrónicas, Electromecánicas y Luminotécnicas.}, respectivamente \citep{CIAA}.

\subsection{Objetivos del proyecto}
\begin{enumerate}
  \item Impulsar el desarrollo tecnológico nacional, a partir de sumar valor agregado al trabajo y a los productos y servicios, mediante el uso de sistemas electrónicos, en el marco de la vinculación de las instituciones educativas y el sistema científico-tecnológico con la industria.
  \item Darle visibilidad positiva a la electrónica argentina.
  \item Generar cambios estructurales en la forma en la que se desarrollan y utilizan en nuestro país los conocimientos en el ámbito de la electrónica y de las instituciones y empresas que hacen uso de ella.
\end{enumerate}

Es importante destacar que La CIAA-NXP es la primera y única computadora del mundo que reúne dos cualidades:

\begin{enumerate}
  \item Ser Industrial, ya que su diseño está preparado para las exigencias de confiabilidad, temperatura, vibraciones, ruido electromagnético, tensiones, cortocircuitos, etc., que demandan los productos y procesos industriales.
  \item Ser Abierta, ya que toda la información sobre su diseño de hardware, firmware, software, etc. está libremente disponible en internet bajo la Licencia BSD, para que cualquiera la utilice como quiera.
\end{enumerate}

\section{Proceso de elaboración del vino}

\subsection{La temperatura}

  Un factor vital para que la fermentación siga correctamente su curso. Es importante saber que, si la temperatura de fermentación es elevada, esos grados de más matarán prematuramente las levaduras (que no podrán completar su función metabólica).

  Pero, además de este aspecto de vital importancia, también es necesario saber que una temperatura excesiva provocará la pérdida de elementos aromáticos y un incremento del amargor (ya que el calor puede desencadenar una reacción química de las partes sólidas procedentes de la vid, como pueden ser los hollejos).

  Por este motivo, la temperatura es uno de los aspectos más vigilados en el proceso de fermentación del vino incluso desde antes de dejar que la naturaleza obre sus misterios químicos





