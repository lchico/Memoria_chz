% Chapter 1

\chapter{Introducción General} % Main chapter title

\label{Chapter1} % For referencing the chapter elsewhere, use \ref{Chapter1} 
\label{IntroGeneral}

%----------------------------------------------------------------------------------------

% Define some commands to keep the formatting separated from the content 
\newcommand{\keyword}[1]{\textbf{#1}}
\newcommand{\tabhead}[1]{\textbf{#1}}
\newcommand{\code}[1]{\texttt{#1}}
\newcommand{\file}[1]{\texttt{\bfseries#1}}
\newcommand{\option}[1]{\texttt{\itshape#1}}
\newcommand{\grados}{$^{\circ}$}

%----------------------------------------------------------------------------------------

\section{Introducción}

%----------------------------------------------------------------------------------------

\section{Producción de vino}


\section{Motivación}

En el presente trabajo se busca satisfacer las necesidades de la bodega de mi primo. La cual se necesita controlar un proceso delicado como es la fermentación del vino y se requiere un cuidado continúo. Por el momento es realizado por personal día y noche. Teniendo en cuenta que en la zona de trabajo es difícil conseguir personas responsables que se comprometan con el trabajo. 

A su vez se busca también una satisfaccion personal, dado que hacía tiempo que buscaba poder implementar en un microtrolador una interfaz que permita visualizar la información de un sistema mediante una plataforma web, donde se pueda apreciar el estado de las entradas y poder ejercer control sobre ellas.

Y para concluir, el hecho de utilizar la plataforma CIAA, la cual me permitió cumplir todo estos objetivos, haciendo uso de un sitema operativo de tiempo real (RTOS), como sé llamará de aqui en adelante.



\section{Proyecto CIAA}

\section{Objetivos y Alcances}





